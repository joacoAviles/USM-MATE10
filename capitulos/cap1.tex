\section{Fundamentos del Lenguaje Matemático}
\subsection{Nociones de lógica y teoría de conjuntos}
\subsubsection{Conectivos Lógicos y Tablas de Verdad}

\begingroup
\setlength{\tabcolsep}{5pt} % Default value: 6pt
\renewcommand{\arraystretch}{1.5} % Default value: 1
\begin{center}
  \begin{tabular}{c|c|cccccc}
              &           & Conjunción         & Disyunción       & Implicación Condicional & Equivalencia Bicondicional  & Negación        & Disyunción exclusiva \\
    {\bf $p$} & {\bf $q$} & {\bf $p \wedge q$} & {\bf $p \vee q$} & {\bf $p \Rightarrow q$} & {\bf $p \Leftrightarrow q$} & {\bf $\bar{p}$} & {\bf $p \veebar q$}  \\\hline
    1         & 1         & 1                  & 1                & 1                       & 1                           & 0               & 0                    \\
    1         & 0         & 0                  & 1                & 0                       & 0                           & 0               & 1                    \\
    0         & 1         & 0                  & 1                & 1                       & 0                           & 1               & 1                    \\
    0         & 0         & 0                  & 0                & 1                       & 1                           & 1               & 0
  \end{tabular}
\end{center}
\endgroup

\subsubsection{Álgebra de Proposiciones}

\begingroup
\setlength{\tabcolsep}{6pt} % Default value: 6pt
\renewcommand{\arraystretch}{1.5} % Default value: 1
\begin{center}
\begin{tabular}{l|c}
\bf Nombre&\bf Propiedad\\ \hline
Identidad& $p \wedge V \equiv p, \hspace{0.3cm} p \wedge F \equiv F, \hspace{0.3cm} p \vee V \equiv V, \hspace{0.3cm} p \vee F \equiv p  $\\
Idempotencia& $p \wedge p \equiv p,\hspace{0.3cm} p \vee p \equiv p $\\
Involución& $\overline{( {\overline{p}} )} \equiv p$\\
Complemento&$p \wedge \bar{p} \equiv F,\hspace{0.3cm} p \vee \bar{p} \equiv V$\\
Conmutatividad&$ p \wedge q \equiv q \wedge p, \hspace{0.3cm} p \vee q \equiv q \vee p $\\
Asociatividad&$ p \wedge (q \wedge r) \equiv (p \wedge q) \wedge r, \hspace{0.3cm} p \vee (q \vee r) \equiv (p \vee q) \vee r$\\
Distributividad&$p \wedge (q \vee r) \equiv (p \wedge q) \vee (p \wedge r), \hspace{0.3cm}  p \vee (q \wedge r) \equiv (p \vee q) \wedge (p \vee r) $\\
Leyes de Morgan&$\overline{p \vee q} \equiv \bar{p} \wedge \bar{q}, \hspace{0.3cm} \overline{p \wedge q} \equiv \bar{p} \vee \bar{q}$\\
Transitividad&$[(p \Rightarrow q) \wedge (q \Rightarrow r)] \Rightarrow (p \Rightarrow r)$\\
Absorción&$[p \wedge ( p \vee q ) \equiv p, \hspace{0.3cm} [p \vee (p \wedge q)] \equiv p$\\
C. de la implicamia& $(p \Rightarrow q) \equiv \bar{p} \vee q$\\
Equivalencia dividada& $(p \Leftrightarrow q) \equiv (p \Rightarrow q) \wedge (q \Rightarrow p)$\\
Por casos& $(p_1 \vee p_2 \vee \dots \vee p_n \Rightarrow q) \equiv (p_1 \Rightarrow q) \wedge (p_2 \Rightarrow q) \wedge \dots \wedge (p_n \Rightarrow q)$
\end{tabular}

\end{center}
\endgroup