\newpage
\section{Conjuntos}
\begin{description}
  \item[Conjunto:] Concepto primitivo, pero puede ser considerado como una colección de elementos u objetos 
\end{description}

\subsection{Definiciones Básicas}

\begingroup
\setlength{\tabcolsep}{5pt} % Default value: 6pt
\renewcommand{\arraystretch}{1.5} % Default value: 1
\begin{center}
  \begin{tabular}{l|l}
    Unión de A y B& $A \cup B = \lbrace x \in \mathcal{U} / x \in A \vee x \in B \rbrace$\\
    Intersección A y B&$A \cap B = \lbrace x \in \mathcal{U} / x \in A \wedge x \in B \rbrace$\\
    Diferencia de A y B&$A - B = \lbrace x \in A / x \notin B \rbrace$\\
    Complemento de A&$A^c = \mathcal{U} - A = \lbrace x \in \mathcal{U} / x \notin A \rbrace$
  \end{tabular}
\end{center}
\endgroup

\subsection{Proposición}
Sean $A, B$ y $C$ conjuntos. Se tiene las siguientes propiedades:
\begingroup
\setlength{\tabcolsep}{5pt} % Default value: 6pt
\renewcommand{\arraystretch}{1.5} % Default value: 1
\begin{center}
\begin{tabular}{l|c}
\bf Nombre&\bf Propiedades\\ \hline
Identidad&$A \cap U = A,\hspace{0.3cm} A \cap \phi = \phi, \hspace{0.3cm} A \cup \mathcal{U} = \mathcal{U}, \hspace{0.3cm} A \cup \phi = A$\\
Idempotencia&$A \cap A = A, \hspace{0.3cm} A \cup A = A$\\
Involución&$(A^c)^c = \A\\
Complemento&$A \cap A^c = \phi, \hspace{0.3cm} A \cup A^c = \mathcal{U}$\\
Conmutatividad&$A \cap B = B \cap A, \hspace{0.3cm} A \cup B = B \cup A $\\
Asociatividad&$ A \cap (B \cap C) = (A \cap B) \cap C, \hspace{0.3cm} A \cup (B \cup C) = (A \cup B) \cup C $\\
Distributividad&$A \cap (B \cup C) = (A \cap B) \cup (A \cap B), \hspace{0.3cm} A \cup (B \cap C) = (A \cup B) \cap (A \cup C)$\\
Leyes de Morgan&$(A \cap B)^c = A^c \cup B^c, \hspace{0.3cm} (A \cup B)^c = A^c \cap B^c$
\end{tabular}
\end{center}
\endgroup

\hrule
$$A \subseteq B \Leftrightarrow B^c \subseteq A^c$$
$$A \subseteq B \Rightarrow A \cap B = A  \hspace{0.1cm} \wedge \hspace{0.1cm}  A \cup B =B$$
$$A \subseteq B \hspace{0.1cm} \wedge \hspace{0.1cm} B \subseteq C \Rightarrow A \subseteq C$$
$$A \cap B \subseteq A \subseteq A \cup B$$
\hrule

\subsubsection{Cardinalidad}
Llamaremos cardinalidad de un conjunto A (denotado por $|A|$ o $\sharp A$), al número de elementos que lo forman. Si no existe un número natural que corresponda al número de elementos de un conjunto A, diremos que el conjunto tiene infinitos elementos, o que A es infinito. En caso contrario, diremos que A es finito.

\begingroup
\setlength{\tabcolsep}{5pt} % Default value: 6pt
\renewcommand{\arraystretch}{1.5} % Default value: 1
\begin{center}
\begin{tabular}{l|c}
\bf Proposición&\bf Propiedades\\ \hline
Sea $A,B \subseteq U$ tal que $|A|, |B| < \infty$&$|A \cup B| = |A| + |B| - |A \cap B|$\\
Sea $A,B$ y $C$ subconjuntos de un universo finito& $|A \cup B \cup C| = |A| + |B| + |C| - |A \cap B| - |A \cap C| - |B \cap C| + |A \cap B \cap C|$\\
Sean $A, B \subseteq U$ tal que $|A|, |B| < \infty$& Si $A \subseteq B$, entonces $|A| \le |B|$\\
&$|A^c| = |u| - |A|$\\
&$|A-B|=|A|-|A \cap B|$
\end{tabular}
\end{center}
\endgroup



%\big(\big)
